\section{Polyphonic Music Prediction System} 
\label{sec:prediction}
Taking inspiration from speech recognition, it has been known that a good statistical model of symbolic music can help the transcription process \cite{Cemgil2004a}. However there are 2 main reasons for the use of MLMs in AMT not being more common. 
\begin{enumerate}
\item Training models that capture the temporal structure and complexity in symbolic polyphonic music is not an easy task. In speech recognition, often simple language models like n-grams work extremely well. However, music has a more complex structure and simple statistical models like n-grams and HMMs fail to model these characteristics accurately ever for music with simple structure\cite{Boulanger-Lewandowski2012}.
\item There is no consensus on how to incorporate this prior information within the transcription system. However, recently there have been some successful attempts at using this prior information to improve the accuracy on AMT tasks \cite{Boulanger-Lewandowski2012,Boulanger-Lewandowski2013}. 

\end{enumerate}


In this section we discuss the details of the music prediction system and the models used. In the next section we discuss how we incorporate the predictions from these models in a PLCA-based music transcription system. 

%It was demonstrated in \cite{Boulanger-Lewandowski2012} how a music language model (MLM) can be used to improve the transcription performance of a purely acoustic model. The MLM employed there was based on the recurrent neural network-restricted Boltzmann machine (RNN-RBM). A related model --- the recurrent neural network-neural autoregressive distribution estimator (RNN-NADE) was also used for the same purpose with comparable results. In the present work, we employ both the standard RNN, and the RNN-NADE as MLMs for boosting the transcription accuracy of the PLCA based model described in the previous section. In this section, we briefly describe the RNN-NADE which we used in our work as the MLM, and the necessary background for understanding this model.

	\subsection{Recurrent Neural Network}
	\label{subsec:rnn}
	A recurrent neural network (RNN) is a powerful model for time-series data which is known to account for long-term temporal dependencies, over multiple time-scales when trained effectively. 
	Given a sequence of inputs $v_1, v_2, \ldots, v_T$ each in $\mathbb{R}^n$, the network computes a sequence of hidden states $\hat{h}_1, \hat{h}_2, \ldots, \hat{h}_T$ each in $\mathbb{R}^m$, and a sequence of predictions $\hat{y}_1, \hat{y}_2, \ldots, \hat{y}_T$ each in $\mathbb{R}^k$ by iterating the equations
	\begin{eqnarray}
		\hat{h}_{t} & = & e(W_{\hat{h}x} v_t + W_{\hat{h}\hat{h}} \hat{h}_{t-1} + b_{\hat{h}}) \\
		\hat{y}_t & = & g(W_{y\hat{h}})
	\end{eqnarray}
	
	where $W_{y\hat{h}}$, $W_{\hat{h}x}$, $W_{\hat{h}\hat{h}}$ are the weight matrices and $b_{\hat{h}}$, $b_y$ are the biases and $e$ and $g$ are activation functions which are typically non-linear and applied element-wise. %The RNN also has a special initial bias $b^{init}_{\hat{h}}$ which replaces the formally undefined expression $W_{\hat{h}\hat{h}} \hat{h}_0$ at time $t = 1$. 
	
	In theory, a recurrent neural network can be trained using the gradient-based Back-Propagation Through Time algorithm \cite{Werbos1990} using the exactly computable error gradients in the network. However, $1^{st}$ order gradient methods fail to correctly train RNNs for many real-world problems. This difficulty has been associated with what is known as the \textit{vanishing/exploding gradients} phenomenon \cite{Bengio1994}, where the errors exhibit exponential decay/growth as they are back-propagated through time. %Several proposals have been made to overcome this difficulty while retaining the predictive power of the RNN \cite{Hochreiter1997, Jaeger2002, Martens2011}. % Optionally, if low on space, replace preceding paragraph with the following sentence and add it to the first paragraph of this RNN subsection: Owing to the difficulty in learning long-term temporal dependencies with the RNN with the gradient-based Back-Propagation Through Time algorithm \cite{Werbos1990}, various alternatives for training RNNs have been proposed over the years \cite{Hochreiter1997, Jaeger2002, Martens2011}.

	However, recent in work in the field of neural networks and deep learning has led to several improvements in gradient based optimization methods that make training of RNNs on real-world data possible. Most notably, the Hessian Free (HF) optimization algorithm has been used to train RNNs successfully on several real world datasets, including symbolic polyphonic music data \cite{Martens2011}. Apart from second order methods like HF, several modifications to first-order gradient based methods exist that currently form the state of the art in training RNNs \cite{bengio2012advances}. 
%
%	\subsection{Restricted Boltzmann Machine}
%	\label{subsec:rbm}
%	A restricted Boltzmann Machine (RBM) is an energy-based model with a bipartite structure consisting of a \textit{visible} layer $v$ (with bias parameters $b_v$), a \textit{hidden} layer $h$ (with bias $b_h$) and a weight-matrix $W_{vh}$ between these two layers \cite{Smolensky1986, Hinton2002}. The joint probability of a given configuration of the visible layer $v$ and hidden layer $h$ is given by $p(v, h) = exp(-h^T W v - b_{v}^{T} v - b_{h}^{T} h) / Z$, where $Z$ is the partition function which is usually intractable. An RBM can be trained in an unsupervised manner using the Contrastive Divergence learning algorithm \cite{Hinton2002, Tieleman2008}. Inference in RBMs involves 	


	\subsection{Recurrent Neural Network-based models}
	\label{subsec:rnn-based}
	One of the drawbacks of using RNNs to predict polyphonic symbolic music is that the outputs of the network, $\hat{y}_T$ at any time step $T$,  are conditionally independent given the sequence of input vectors $v_1, v_2, \ldots, v_T$. This is a severe constraint when used for modelling polyphonic music, where notes often appear in very correlated patterns within a frame. In order to overcome this limitation, models derived from RNNs have been proposed which are better at modelling high-dimensional sequences \cite{sutskever2008recurrent,Boulanger-Lewandowski2012}.

	The first RNN-based model that tried to model high-dimensional sequences is the Recurrent Temporal Restricted Boltzmann Machine (RTRBM) \cite{sutskever2008recurrent}. This model was extended to the more general RNN-RBM model, where the hidden states for the RBM and RNN were not constrained to be the same. For our prediction system, we make use of a variant of the RNN-RBM, called the RNN-NADE. The only difference is that the conditional distributions at each step are modelled by a Neural Autoregressive Distribution Estimator (NADE) \cite{Larochelle2011} as opposed to an RBM. As discussed in the next section, to combine the predictions with the transcription system, we need individual pitch activation probabilities at each time-step. Obtaining these probabilities from an RBM is intractable as it requires summing over all possible hidden states. However the NADE is a tractable distribution estimator and we can easily obtain these probabilities from the NADE. The NADE models the probability of occurrence of a 
vector $p$ as: 
	\begin{equation}
		P(p) = \prod_{i=1}^D P(p_i|\mathbf{p_{<i}})
	\end{equation}
	where $ p \in \mathbb{R}^{D}$,$p_{i}$ is the pitch activation and $\mathbf{p_{<i}}$ is the vector containing all the pitch activations $p_j$ such that $j<i$. 
	
	In our system we utilise each of the conditional probabilities $p(p_i|\mathbf{p_{<i}})$ as probabilities of the pitch activations. Although the pitch activation probabilities are only conditioned on $\mathbf{p_{<i}}$, we hypothesize that this will be a better model than the RNN, where the pitch activation probabilities are completely independent. Another motivation for using the NADE is that the gradients can be computed exactly, and therefore we can employ HF optimization for training the RNN-NADE. 

	%The neural autoregressive distribution estimator (NADE) \cite{Larochelle2011} is a graphical model inspired by the Restricted Boltzmann Machine \cite{Smolensky1986, Hinton2002}. It shares the structural properties of the RBM in that it has a visible layer $v$ (with biases $b_v$), a hidden layer $h$ (with biases $b_h$), with these two layers connected by a weight-matrix $W$. It facilitates the exact inference $p(v)$ given an input vector $v$, which is not possible in RBMs since there one has to compute the intractable \textit{partition function} \cite{Larochelle2011}. This was made possible by thinking of the RBM as a \textit{fully visible sigmoid belief network} (FVSBN) \cite{Neal1992}. The FVSBN is a special case of a family of models known as fully visible Bayesian networks \cite{Frey1998} with the property
	%\begin{equation}
	%	p(v) = \prod_{i=1}^D p(v_i|v_{parents(i)})
	%\end{equation}
	%where all observation variables $v_i$ are arranged into a directed acyclic graph and $v_{parents(i)}$ corresponds to all the variables in v that are parents of $v_i$. In an FVSBN, the acyclic graph is obtained by defining the parents of $v_i$ as all variables that are to its left, or $v_{parents(i)} = v_{<i}$ where $v_{<i}$ refers to the subvector containing all variables $v_j$ such that $j<i$. In the case of the NADE, $p(v_i|v_{parents(i)})$ can be computed as follows
	%\begin{eqnarray}
    %		p(v_i=1|v_{parents(i)}) & = & \sigma(b_{v}^{(i)}) + (W^T)_{i,\cdot} h_i) \\
	%	h_i & = & \sigma(b_h + W_{\cdot , <i} v_{<i})
	%\end{eqnarray}
	
	%Untying the weights $W$ and $W^T$ results in a more powerful model. In the NADE, the cost of computing $p(v)$ is $O(HD)$, where $H$ is the number of hidden units and $D$ is the dimensionality of the vector $v$.
	
	%\subsection{Training}
	%\label{subsec:training}
	%We trained the RNN to predict the next frame of symbolic music given a sequence. We trained the RNN using both gradient descent and HF optimization. The models were trained on sequences of length 200. For the NADE we employed a similar strategy and trained it using both gradient descent and HF optimization. The models were trained separately on data from the Nottingham dataset and the JSB Chorales dataset. We validate our models, by computing prediction accuracies as described in the next section. 
	


% Models for music prediction capture structure in musical data. Such models deal with music data in the symbolic form \cite{Orio2006}. Structure in music can be both sequential and parallel. For example, a melody can be thought to have purely sequential structure since . On the other hand, a polyphonic piece of music has parallel structure as well due to the added possibility of occurrence of simultaneous pitches at any given time instant.